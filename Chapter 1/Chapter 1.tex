\documentclass{article}

\usepackage{ctex}
\usepackage{amsmath}
\usepackage[a4paper, left=2cm, right=2cm, top=2.5cm, bottom=2.5cm]{geometry}


\pagestyle{plain}

\begin{document}

\zihao{5}

\title{第一章 \ 经济学原理}
\author{Treap}
\date{\today}

\maketitle

经济学是研究资源配置的一门社会科学。作为 Web3 行业的参与者,我们其实都具有不同程度的经济思维,也总是在不经意间用经济学的方法来思考问题。假如你有 100 万美元,你会拿出多少钱去消费,多少钱去投资加密货币?如果你投资加密货币,你的持仓分配是怎样的?你的投资目标或者预期收益是多少?市场上还有千千万万像你一样的人,如果把你们的行为加总起来,整个市场会变成什么样子?

虽然你可能暂时还没有 100 万美元,但是这并不妨碍你用一种更加本质化的方式去感知、去思考这个纷繁复杂的加密市场。幸运的是,在我们之前的几百年,已经有一批又一批的巨人为我们提供了肩膀。在享用他们伟大的思维成果时,我们可以带着批判性的眼光去检阅这些理论的科学性与适用性,尤其是在 Web3 世界的全新经济秩序下。

\section{经济学的基本假设}

物理学家经常假设他们的研究对象是球形的,没有摩擦阻力,不会发生形变。这种假设是为了简化分析过程,剥离不必要的表象,深入问题的本质。同样,经济学家也需要引入一些基本的假设,剥离个性化的特征,找出一种具有代表性的行为模式。

\subsection{理性行为假设}

《史记·货殖列传》有言:“天下熙熙,皆为利来;天下攘攘,皆为利往。”

利己主义

\textbf{在经济活动中,主体所追求的唯一目标是自身经济利益的最大化。}

\section{供给、需求与均衡}

\section{宏观经济学基础}

\end{document}